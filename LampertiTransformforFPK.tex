\documentclass[11pt,twoside,a4paper]{article}
\usepackage{amsmath,amssymb,mathrsfs,euscript,yfonts,psfrag,latexsym,dsfont,graphicx,bbm,color,amstext,wasysym,subfig,flushend,parskip,url,soul,bm,cite,balance}
\usepackage{amsthm}
\usepackage{algorithm}
\usepackage{algpseudocode}
\usepackage{algorithmicx}
\usepackage[symbol]{footmisc}
%\usepackage{showkeys}
%\usepackage{epstopdf,hyperref,pdfsync,url}
\usepackage[dvipsnames]{xcolor}
\usepackage[makeroom]{cancel}
\graphicspath{{./},{./figures/}}

\DeclareMathOperator*{\argmax}{arg\:max}
\DeclareMathOperator*{\argmin}{arg\:min}
\DeclareMathOperator*{\arginf}{arg\:inf}

\begin{document}
\newtheorem{thm}{Theorem}
\newtheorem{corollary}[thm]{Corollary}
\newtheorem{conj}[thm]{Conjecture}
\newtheorem{lemma}[thm]{Lemma}
\newtheorem{proposition}[thm]{Proposition}
\newtheorem{problem}{Problem}
\newtheorem{remark}{Remark}
\newtheorem{definition}{Definition}
\newtheorem{example}{Example}
\newcommand{\pinf}{\rho_{\infty}}
\newcommand{\prox}{\rm{prox}}
\newcommand{\bp}{{\bm{p}}}
\newcommand{\bq}{{\bm{q}}}
\newcommand{\bmm}{{\bm{m}}}
\newcommand{\bc}{{\bm{c}}}
\newcommand{\be}{{\bm{e}}}
\newcommand{\bps}{{\bm{\psi}}}
\newcommand{\br}{{\bm{r}}}
\newcommand{\bw}{{\bm{w}}}
\newcommand{\bx}{{\bm{x}}}
\newcommand{\bxx}{{\bm{X}}}
\newcommand{\by}{{\bm{y}}}
\newcommand{\bz}{{\bm{z}}}
\newcommand{\bth}{{\bm{\theta}}}
\newcommand{\bg}{{\bm{\gamma}}}
\newcommand{\bsim}{{\bm{\sigma}}}
\newcommand{\bom}{{\bm{\omega}}}
\newcommand{\bxi}{{\bm{\xi}}}
\newcommand{\bet}{{\bm{\eta}}}
\newcommand{\bzz}{{\bm{Z}}}
\newcommand{\bbf}{{\bm{f}}}
\newcommand{\brh}{{\bm{\varrho}}}
\newcommand{\balpha}{{\bm{\alpha}}}
\newcommand{\bbeta}{{\bm{\beta}}}  .   
\newcommand{\bdelta}{{\bm{\delta}}}
\newcommand{\bPi}{{\bm{\Pi}}}
\newcommand{\bE}{{\mathbb{R}}}
\newcommand{\bC}{{\bm{C}}}
\newcommand{\bL}{{\bm{L}}}
\newcommand{\diff}{{\rm{d}}}
\newcommand{\diag}{{\rm{diag}}}
\newcommand{\bI}{{\bm{I}}}
\newcommand{\bP}{{\bm{P}}}
\newcommand{\bR}{{\bm{R}}}
\newcommand{\bD}{{\bm{D}}}
\newcommand{\bQ}{{\bm{Q}}}
\newcommand{\ptwo}{\mathcal{P}_{2}(\Omega)}
\newcommand{\epin}{{ \epsilon^{-1}}}
\renewcommand{\dagger}{{T}}






\renewcommand{\thefootnote}{\fnsymbol{footnote}}



\newcommand{\differential}{{\rm{d}}}

\newcommand{\interior}[1]{%
  {\kern0pt#1}^{\mathrm{o}}%
}

\newcommand{\bR}{\mathbb{R}}
\newcommand{\diag}{\operatorname{diag}}
\newcommand{\tr}{\operatorname{trace}}
%\renewcommand{\odot}{\circ}
\newcommand{\ignore}[1]{}

\newcommand{\magenta}{\color{magenta}}
\newcommand{\red}{\color{red}}
\newcommand{\blue}{\color{blue}}
\newcommand{\gray}{\color{gray}}
\definecolor{grey}{rgb}{0.6,0.3,0.3}
\definecolor{lgrey}{rgb}{0.9,.7,0.7}


%\renewcommand{\qedsymbol}{\hfill\ensuremath{\blacksquare}}

\newcommand{\symsum}{\displaystyle\sum_{\rm{symm}}}

\def\spacingset#1{\def\baselinestretch{#1}\small\normalsize}
\setlength{\parindent}{20pt}
\setlength{\parskip}{12pt}
\spacingset{1}

\title{\huge{Lamperti Transform For Power Systems
}}

\author{Kenneth Caluya}

\markboth{\today}{}

\maketitle
We consider the system of 2nd-order SDEs given by
\begin{align*}
    m_i\ddot{\theta}p
    +  \gamma_i \dot{\theta}_i = 
P_i - \sum_{j=1}^{n} k_{ij} \sin(\theta_i-\theta_j)+ \sigma_i \times \:\text{stochastic forcing}\: i=1,\dots,n.
\end{align*}
We can transform the system into 
\begin{align} \label{ODEsystem}
    \diff \bth &= \bom \: \diff t \nonumber \\ 
    \diff \bom & = -\diag(\bg \oslash \bmm)\: \bom-\nabla_{\bth} V(\bth) + \diag(\bsim \oslash \bmm) \: \diff \bw
\end{align}
where $\bth=(\theta_1,\theta_2,\dots,\theta_n)$ and $\bom=(\dot{\theta}_1,\dot{\theta}_2,\dots,\dot{\theta}_n)$ and the potential function is given by 
\begin{align}
    V(\theta):= \sum_{i=1}^{n} \frac{1}{m_i}P_i \theta_i + \sum_{(i,j)\in \mathcal{E}}\frac{1}{m_i}k_{ij}(1-\cos(\theta_i-\theta_j))
\end{align}
\begin{thm}[Lamperti Transform] \label{LampertiTransform} Let $X_t$ be the solution to the (Ito) SDE 
\begin{align}
    \diff \bxx_t = \bbf(\bxx_t,t) \: 
    \diff t + \bbeta(\bxx_t,t) \: \bR(t)\diff \bm{W}_t
\end{align}
where $\bxx_t,w_t \in \mathbb{R}^d$, $\bR(t)\in \mathbb{R}^{d \times d}$ is any matrix function of $t$ and $\bbeta \in \mathbb{R}^{d \times d} $ is a diagonal matrix whose diagonal elements which we denote 
\begin{align*}
    \beta_{i,i}(\bxx_t.t)= \beta_i(X_{i,t},t),
\end{align*}
i.e., each diagonal element depend only on the $i$th component of $\bxx_t$. Then, the transformation defined by 
\begin{align} \label{ChangeofVariable}
Z_{i,t} = \psi_i(X_{i,t},t) = \int \frac{1}{\beta_i(x,t)} \: \diff x \bigg |_{x=X_{i,t}}
\end{align}
will result in the diffusion process 
\begin{align} \label{CoVDiffusion}
    \diff Z_{i,t} = \left(\frac{\partial}{\partial t}\psi_i(x,t)\bigg|_{x=X_{i,y}} + \frac{f_i(\bps^{-1}(\bzz,t).t)}{\beta_i(\psi_i^{-1}(Z_{i,t},t),t)} +\frac{1}{2}  \frac{\partial}{\partial x}
    \beta_i(\psi_i^{-1}(Z_{i,t},t),t) \right)\diff t + \sum_{j=1}^{d}
 r_{i,j}(t)\: \diff w_{j,t}
\end{align}
where $r_{i,j}(t)$ are the elements of $\bR(t)$ and $\bxx_t = \psi^{-1}(\bzz,t)$.
\end{thm}

\begin{proposition}
Let $d=2n$ and set 
\begin{align} \label{partitionvector}
     \bxx_t = \begin{bmatrix}\bth \\ 
    \bom
    \end{bmatrix}, \quad  \bzz_t := \begin{bmatrix}
    \bxi \\
    \bet 
    \end{bmatrix}.
\end{align}
Let 
\begin{align}
    \bbeta(\bxx,t) \equiv   \begin{bmatrix}
    \bI & 0 \\ 
    0 & \diag(\bsim \oslash \bmm )
    \end{bmatrix} ,
    \quad 
    \bR(t) \equiv 
     \begin{bmatrix}
    0 & 0 \\ 
    0 & \bI
    \end{bmatrix} 
\end{align}
then applying Theorem \ref{LampertiTransform} to (\ref{ODEsystem}) results in the diffusion process
\begin{align}
    \diff \bz_t
\end{align}
\end{proposition}
\begin{proof}
Notice that performing the change of variable (\ref{ChangeofVariable})
results in 
\begin{align}
    \bzz_t = \bps(\bxx,t) = 
     \begin{bmatrix}
    \bI & 0 \\ 
    0 & \diag(\bmm \oslash \bsim )
    \end{bmatrix} \bxx_t
\end{align}
so that 
\begin{align}
    \bxx_t = \bps^{-1}(\bzz,t) =
      \begin{bmatrix}
    \bI & 0 \\ 
    0 & \diag(\bsim \oslash \bmm )
    \end{bmatrix}
    \bzz_t.
\end{align}
Using (\ref{partitionvector}), the inverse transformation can be written component wise as 
\begin{align} \label{compontentwisetransform}
    \theta_i = \xi_i,\quad \omega_i = \frac{\sigma_i}{m_i}  \eta_i
\end{align}
Notice that since $\bps$ and $\bbeta$ are independent of $t$ and $x$, then the first and third term in the drift of (\ref{CoVDiffusion}) vanish and we are left with the second term. Notice that we have
\begin{align}
  \frac{f_i(\bxx_t,t)}{\beta_i(\bxx_{i,t},t)} = \begin{cases}
  \omega_i & 1\leq i \leq n \\ 
  -\frac{\gamma_i}{m_i}\frac{m_i}{\sigma_i} \omega_i -\frac{m_i}{\sigma_i}\frac{P_i}{m_i} - \displaystyle \sum_{j=1}^{n} \frac{m_i}{\sigma_i} \frac{1}{m_i}k_{ij} \sin(\theta_i-\theta_j) & n+1\leq i \leq 2n
  \end{cases}
\end{align}
which means that 
\begin{align}
    \frac{f_i(\bps(\bzz,t),t)}{\beta_i(\psi(Z{i,t},t),t)} = \begin{cases}
    \frac{\sigma_i}{m_i} \eta _i & 1\leq i \leq n \\ 
    -\frac{\gamma_i}{m_i}\eta_i-\frac{P_i}{\sigma_i} -\displaystyle \sum_{j=1}^{n}\frac{1}{\sigma_i} k_{i,j}\sin(\xi_i-\xi_j) & n+1\leq i \leq 2n
    \end{cases}.
\end{align}
This results in the diffusion process 
\begin{align}
\diff \begin{bmatrix}
 \bxi \\ \bet 
\end{bmatrix}
&= 
\begin{bmatrix}
\diag(\bsim \oslash \bmm) \bet \\
- \diag(\bg \oslash \bmm) \bet- \nabla_{\bxi} U(\bxi) 
   \end{bmatrix}
   \: \diff t
   + \begin{bmatrix}
    0 & 0 \\ 
    0 & \bI
    \end{bmatrix} \: \diff \bm{W}_t \\ 
    :&=\begin{bmatrix}
 \bm{G}_1 \bet \\
 -\nabla _{\bet} F(\bet)  - \nabla_{\bxi} U(\bxi)
\end{bmatrix} \: \diff t   + \begin{bmatrix}
    0 & 0 \\ 
    0 & \bI
    \end{bmatrix} \: \diff \bm{W}_t
\end{align}
where $F(\bet) =\frac{1}{2} \bet^{\top} \bm{G}_2 \bet$
where the potential function $U$ is given by 
\begin{align}
U(\bxi):= \sum_{i=1}^{n} \frac{P_i}{\sigma_i} \xi_i - \sum_{i,j\in \mathcal{E}}\frac{1}{\sigma_i} k_{i,j}(1-\cos(\xi_i-\xi_j))\end{align}
\end{proof}

The Fokker-Planck equation of this SDE is 
\begin{align}
    \frac{\partial \rho}{\partial t} = -\langle \bm{G}_1\bet, \nabla_{\bxi} \rho\rangle + \nabla_{\bet} \cdot \left( \left( \nabla _{\bet}F(\bet)  + \nabla_{\bxi} U(\bxi)
    \right) \rho\right) + \Delta_{\bet} \rho
 \end{align}

\subsection{Stationary Solution}
To compute the stationary solution, we set 
\begin{align} \label{EqualZeroPDE}
    0 = \langle \bm{G}_1\bet, \nabla_{\bxi} \rho\rangle + \nabla_{\bet} \cdot \left( \left( \nabla _{\bet}F(\bet)  + \nabla_{\bxi} U(\bxi)
    \right) \rho\right) + \Delta_{\bet} \rho
\end{align}
and we have the ansatz 
\begin{align}
    \rho_{\infty} \propto \exp\left(-(F(\bet)+U(\bxi)\right).
\end{align}
From this, we get the derivatives 
\begin{align} \label{derivatives}
    \nabla_{\bxi} \rho_{\infty} = -\pinf \nabla_{\bxi}U(\bxi) ,
    \quad \nabla_{\bet} \pinf = -\pinf \nabla_{\bet} F(\bet).
\end{align}
Using (\ref{derivatives}), we obtain 
\begin{align}
    \Delta_{\eta}\pinf = \pinf \lVert G_2 \eta \rVert^2 - \pinf {\rm{tr}}(G_2) 
\end{align}
and 
\begin{align}
      \nabla_{\bet} \cdot \left( \left( \nabla _{\bet}F(\bet)  + \nabla_{\bxi} U(\bxi)
    \right)\right)= -\pinf \lVert G_2 \eta \rVert^2 - \pinf \langle G_2\bet , \nabla_{\bxi}U(\bxi) + \pinf {\rm{tr}}(G_2)
\end{align}t
which implies that (\ref{EqualZeroPDE}) reduces to 
\begin{align}
    0 = \pinf \langle (G_2-G_1)\bet, \nabla_{\bxi} U(\bxi) \rangle.
\end{align}
In order for this to hold for all $\bet,\bxi$ we must have $G_2=G_1$ which means that $\bsim=\bg$, i.e., $\sigma_i=\gamma_i$ for all $i$.

\subsection{Free Energy}
Consider the functional 
\begin{align}
    \Phi(\rho)& =\int \rho(\bet,\bxi) \log  \rho(\bet,\bxi) \: \diff \bet \diff \bxi  + \beta
    \int \rho(\bet,\bxi) U(\bxi)\: \diff \bet \diff \bxi  
    +
   \beta \int \rho(\bet,\bxi) F(\bet) \: \diff \bet \diff \bxi \nonumber \\ 
   & = \beta \widetilde{\Phi}(\rho) 
\end{align}
We compute 
\begin{align}
    \frac{\diff \Phi}{\diff t}&=\int \frac{\partial \rho}{\partial t} \left( \log \rho + \beta F +  \beta U\right)\: \diff \bet \diff \bxi \: \diff \bet \diff \bxi    \nonumber \\
    &  = \int \nabla_{\bet} \cdot \left( \rho \left( \nabla_{\bet} \log \rho+ \nabla_{\bet} F + \nabla_{\bxi }U\right)\right) \left( \log \rho + \beta F +  \beta U\right) \: \diff \bet \diff \bxi   \nonumber \\ 
    & \: \: -  \int \langle G_1\bet , \nabla_{\bxi} \rho \rangle \left( \log \rho + \beta F + \beta U\right)\: \diff \bet \diff \bxi   \nonumber \\ 
    & = \int - \lvert \nabla_{\bet} \log \rho \lvert^2\rho  -( \beta+1) \langle \nabla_{\bet} \log \rho ,\nabla_{\eta}F\rangle\rho    - \beta \lvert \nabla_{\bet} F \rvert^2\rho \: \diff \bet \diff \bxi   \nonumber \\ 
    &  \int -\langle \nabla_{\bxi} U , \nabla_{
    \bet} \log \rho \rangle\rho  -\beta \langle \nabla_{\bxi} U , \nabla_{
    \bet} F \rangle\rho  + \langle G_1 \bet, \nabla_{\bxi} \log \rho\rangle \rho  + \beta  \langle G_1 \bet, \nabla_{\bxi} U \rangle\rho  \: \diff \bet \diff \bxi   \nonumber  \\
    & = \int - \lvert \nabla_{\bet} \log \rho \lvert^2\rho  - (\beta+1) \langle \nabla_{\bet} \log \rho ,\nabla_{\eta}F\rangle\rho    - \beta \lvert \nabla_{\bet} F \rvert^2\rho \: \diff \bet \diff \bxi   \nonumber \\
    &+t  \int -\beta \langle \nabla_{\bxi} U , \nabla_{
    \bet} F \rangle\rho +  \beta  \langle G_1 \bet, \nabla_{\bxi} U \rangle\rho. \nonumber 
\end{align}
A sufficient condition for $  \dfrac{\diff \Phi}{\diff t}\leq 0$ is to set  $G_!\bet =\nabla_{\bet} F = G_2 \bet \Rightarrow 
\bsim = \bg \Rightarrow \sigma_i=\gamma_i$ for all $i$.
\end{document}

