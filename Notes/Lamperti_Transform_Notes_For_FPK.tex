\documentclass[11pt,twoside,a4paper]{article}
\usepackage{amsmath,amssymb,mathrsfs,euscript,yfonts,psfrag,latexsym,dsfont,graphicx,bbm,color,amstext,wasysym,subfig,flushend,parskip,url,soul,bm,cite,balance}
\usepackage{amsthm}
\usepackage{algorithm}
\usepackage{algpseudocode}
\usepackage{algorithmicx}
\usepackage[symbol]{footmisc}
%\usepackage{showkeys}
%\usepackage{epstopdf,hyperref,pdfsync,url}
\usepackage[dvipsnames]{xcolor}\graphicspath{{./},{./figures/}}

\DeclareMathOperator*{\argmax}{arg\:max}
\DeclareMathOperator*{\argmin}{arg\:min}
\DeclareMathOperator*{\arginf}{arg\:inf}

\begin{document}
\newtheorem{thm}{Theorem}
\newtheorem{corollary}[thm]{Corollary}
\newtheorem{conj}[thm]{Conjecture}
\newtheorem{lemma}[thm]{Lemma}
\newtheorem{proposition}[thm]{Proposition}
\newtheorem{problem}{Problem}
\newtheorem{remark}{Remark}
\newtheorem{definition}{Definition}
\newtheorem{example}{Example}
\newcommand{\pinf}{\rho_{\infty}}
\newcommand{\prox}{\rm{prox}}
\newcommand{\bp}{{\bm{p}}}
\newcommand{\bq}{{\bm{q}}}
\newcommand{\bmm}{{\bm{m}}}
\newcommand{\bc}{{\bm{c}}}
\newcommand{\be}{{\bm{e}}}
\newcommand{\bps}{{\bm{\psi}}}
\newcommand{\br}{{\bm{r}}}
\newcommand{\bw}{{\bm{w}}}
\newcommand{\bx}{{\bm{x}}}
\newcommand{\bxx}{{\bm{X}}}
\newcommand{\by}{{\bm{y}}}
\newcommand{\bz}{{\bm{z}}}
\newcommand{\bth}{{\bm{\theta}}}
\newcommand{\bg}{{\bm{\gamma}}}
\newcommand{\bsim}{{\bm{\sigma}}}
\newcommand{\bSim}{{\bm{\Sigma}}}
\newcommand{\bGam}{{\bm{\Gamma}}}
\newcommand{\bom}{{\bm{\omega}}}
\newcommand{\bxi}{{\bm{\xi}}}
\newcommand{\bet}{{\bm{\eta}}}
\newcommand{\bzz}{{\bm{Z}}}
\newcommand{\bbf}{{\bm{f}}}
\newcommand{\brh}{{\bm{\varrho}}}
\newcommand{\balpha}{{\bm{\alpha}}}
\newcommand{\bbeta}{{\bm{\beta}}}  .   
\newcommand{\bdelta}{{\bm{\delta}}}
\newcommand{\bPi}{{\bm{\Pi}}}
\newcommand{\bE}{{\mathbb{R}}}
\newcommand{\bC}{{\bm{C}}}
\newcommand{\bL}{{\bm{L}}}
\newcommand{\diff}{{\rm{d}}}
\newcommand{\diag}{{\rm{diag}}}
\newcommand{\bI}{{\bm{I}}}
\newcommand{\bP}{{\bm{P}}}
\newcommand{\bG}{{\bm{G}}}
\newcommand{\bR}{{\bm{R}}}
\newcommand{\bD}{{\bm{D}}}
\newcommand{\bQ}{{\bm{Q}}}
\newcommand{\bM}{{\bm{M}}}
\newcommand{\ptwo}{\mathcal{P}_{2}(\Omega)}
\newcommand{\epin}{{ \epsilon^{-1}}}
\renewcommand{\dagger}{{T}}w
\newcommand{\tr}{{\rm{trace}}}






\renewcommand{\thefootnote}{\fnsymbol{footnote}}



\newcommand{\interior}[1]{%
  {\kern0pt#1}^{\mathrm{o}}%
}

%\renewcommand{\odot}{\circ}
\newcommand{\ignore}[1]{}

\newcommand{\magenta}{\color{magenta}}
\newcommand{\red}{\color{red}}
\newcommand{\blue}{\color{blue}}
\newcommand{\gray}{\color{gray}}
\definecolor{grey}{rgb}{0.6,0.3,0.3}
\definecolor{lgrey}{rgb}{0.9,.7,0.7}


%\renewcommand{\qedsymbol}{\hfill\ensuremath{\blacksquare}}

\newcommand{\symsum}{\displaystyle\sum_{\rm{symm}}}

\def\spacingset#1{\def\baselinestretch{#1}\small\normalsize}
\setlength{\parindent}{20pt}
\setlength{\parskip}{12pt}
\spacingset{1}

\title{\huge{Lamperti Transform For Power Systems
}}

\author{Kenneth Caluya}

\markboth{\today}{}

\section{Gradient Drift Case}
\begin{align}
    \diff \bx = -\nabla \psi (\bx) \diff t + \sqrt{2\beta^{-1}} dw
\end{align}
\begin{align} \label{FPk}
    \frac{\partial \rho}{\partial t} = \nabla \cdot (\rho \nabla \psi) + \beta^{-1}\Delta \rho 
\end{align}



We claim that 
\begin{align}
    F(\rho) &= \mathbb{E}_{\rho}[\psi + \beta^{-1} \log \rho ] \nonumber \\ 
    & = \int \psi(\bx) \rho(\bx) \: \diff \bx + \beta^{-1}\int \rho(\bx) \log \rho(\bx) \: \diff \bx \nonumber \\
    & = \beta^{-1}\int \rho \log \rho-\rho \log( \exp (-\beta \psi)) \: \diff \bx \nonumber \\ 
    & = \beta^{-1} \int \rho \log \left( \frac{\rho}{\exp(-\beta \psi)} \right) \: \diff \bx  \nonumber \\ 
    & =  \beta^{-1} \int \rho \log \left( \frac{\rho}{\frac{1}{Z}\exp(-\beta \psi)} \right) \: \diff \bx  + C  \nonumber 
\end{align}
is a Lyapunov functional along the trajectories of (\ref{FPk}) and $Z= \int\exp(-\beta \psi) \: \diff \bx  $ is a normalization constant.
If you have 
\begin{align}
    \dot{\bx} = \bm{\phi}(\bx)
\end{align}
\begin{align}
    \frac{\diff V}{\diff t } = \langle \nabla V , \dot{x}\rangle =  \langle \nabla V , \bm{\phi}(\bx) \rangle  \leq 0
    \end{align} 
\begin{align}
\frac{\diff F}{\diff t} &= \langle \frac{\delta F}{\delta \rho }, \frac{\partial \rho  }{\partial t}\rangle \nonumber \\ 
& = \langle  \frac{\delta F}{\delta \rho }, \nabla \cdot (\rho \nabla \psi) + \beta^{-1}\Delta \rho  \rangle \nonumber  \\
& = \int  \frac{\delta F}{\delta \rho } (\nabla \cdot (\rho \nabla \psi) + \beta^{-1}\Delta \rho) \: \diff \bx \nonumber \\
& = \int (\psi(\bx) + \beta^{-1}(\log \rho(\bx) +1)) (\nabla \cdot (\rho \nabla \psi) + \beta^{-1}\Delta \rho)\: \diff \bx  \nonumber \\
& = \int (\psi(\bx) + \beta^{-1}(\log \rho(\bx) +1)) (\nabla \cdot (\rho ( \nabla \psi + \beta^{-1}\nabla \log \rho)))\: \diff \bx \nonumber \\
&= - \int  \langle \nabla (\psi(\bx) + \beta^{-1} \log \rho (\bx)), \nabla \psi(\bx) + \beta^{-1} \nabla \log \rho  \rangle \rho \: \diff \bx \nonumber \\
& = -\int \lVert \nabla (\psi(\bx) + \beta^{-1} \log \rho )\rVert^2 \rho(\bx) \: \diff \bx  \nonumber  \\
&=- \mathbb{E}_{\rho}[\lVert \nabla \zeta \rVert^2] \leq 0
 \end{align}
 where $\zeta = \psi + \log \rho $.
\\ 
We consider the system of 2nd-order SDEs given by
\begin{align*}
m_i\ddot{\theta}_i
    +  \gamma_i \dot{\theta}_i = 
P_i - \sum_{j=1}^{n} K_{ij} \sin(\theta_i-\theta_j+\varphi_{ij})+ \sigma_i \times \:\text{stochastic forcing}\: i=1,\dots,n.
\end{align*}
We can transform the system into 
\begin{align} \label{ODEsystem}
    \diff \bth &= \bom \: \diff t \nonumber \\ 
   \bM \diff \bom & = (-\Gamma \bom-\nabla_{\bth} V(\bth)) \diff t + \bSim \: \diff \bw
\end{align}
where $\bth=(\theta_1,\theta_2,\dots,\theta_n), \bom=(\dot{\theta}_1,\dot{\theta}_2,\dots,\dot{\theta}_n), \bM = \diag(m_1,m_2,\dots,m_n), \bm{\Gamma}= \diag(\gamma_1,\gamma_2,\dots,\gamma_n ),\bSim = \diag(\sigma_1,\sigma_2,\dots,\sigma_n)$ and the potential function is given by 
\begin{align}
    V(\bth):= -\sum_{i=1}^{n} P_i \theta_i + \sum_{(i,j)\in \mathcal{E}}k_{ij}(1-\cos(\theta_i-\theta_j+\varphi_{ij}))
\end{align}
we can rewrite as 
\begin{align}
    \diff \underbrace{\begin{bmatrix}
    \bth \\ \bom 
    \end{bmatrix}}_{\bx}
    = \underbrace{\begin{bmatrix}
    \bom  \\ 
 -\bM^{-1}\bm{\Gamma} \bom - \bM^{-1} \nabla_{\bth} V(\bth)
    \end{bmatrix}}_{\bm{f}} \diff t + 
    \underbrace{\begin{bmatrix}
    \bm{0}_{n\times n}  &  \bm{0}_{n\times n}  \\ 
   \bm{0}_{n\times n}  &\bm M^{-1}\bSim  
    \end{bmatrix}}_{\bm{g}}
    \diff \bm{w}
    \end{align}
\subsection{Fokker Planck}
\begin{align}
    \frac{\partial \rho}{\partial t} &= - \nabla_{\theta,\omega} \cdot \left( \rho\begin{bmatrix}
    \bom  \\ 
 -\bM^{-1}\bm{\Gamma} \bom - \bM^{-1} \nabla_{\bth} V(\bth)
    \end{bmatrix} \right) + \frac{1}{2}\sum_{i,j=1}^{n} \frac{\partial^2}{\partial \omega_{i}\omega_j } (\rho \bM^{-1}\Sigma \Sigma^{\top} \bM^{-1} )_{i,j} \nonumber \\
     &= -\nabla_{\theta} \cdot( \rho \bom) + \nabla_{\bom} \cdot \left( \rho (\bM^{-1}\bm{\Gamma}\bom + \bM^{-1}\nabla_{\bth}V(\bth) )\right) + \nabla_{\bom} \cdot \left( \rho D \nabla_{\bom} \log \rho \right) \nonumber \\
     &= - \langle \bom ,\nabla_{\bth}\rho \rangle  + \nabla_{\bom} \cdot \left( \rho \left( \bM^{-1}\bm{\Gamma}\bom + \bM^{-1}\nabla_{\bth}V(\bth) +\bD \nabla_{\bom} \log \rho \right) \right)
\end{align}
where $\bD=\frac{1}{2} \bM^{-1}\bSim \bSim^{\top}\bM^{-1}$.


\begin{thm} {(Ito's Lemma for Multi-Dimensional Processes)} \label{multivariateItoLemma}
Let 
\begin{align}
    \diff \bxx_t = \bbf(\bxx_t,t) \: \diff t + \bm{g}(\bxx_t,t) \diff\bm{w}_t
\end{align}
with $\bxx_t,\bm{w}_t,\bbf(\cdot,) \in \mathbb{R}^n$ and $\bm{g}(\cdot) \in \mathbb{R}^{n \times n}$. Then for a given transformation 
\begin{align}
    \bzz_t = \bm{\psi}(\bxx_t,t) =[\psi_1(\bxx_t,t),\dots,\psi_n(\bxx_t,t)]
\end{align}
where $\bps$ is a function from $\mathbb{R}^n\times [0,\infty]$ into $\mathbb{R}^n$, then $\bzz_t$ is again an Ito process given by 
\begin{align} \label{Itorule}
    \diff Z_{k,t} &= \frac{\partial \psi_k}{\partial t}(\bxx_t,t) \: \diff t + \sum_{i=1}^n \frac{\partial \psi_k}{\partial x_i}(\bxx_t,t) \diff X_{i,t} \nonumber  \\ 
    &+ \frac{1}{2} \sum_{i=1}^n \sum_{j=1}^n \frac{\partial^2  \psi_k}{\partial x_i \partial x_j}(\bxx_t,t) \diff X_{j,t} \diff X_{i,t} 
\end{align}
\end{thm}
\begin{thm}
Let $n=2d$ and consider the transformation defined by 
\begin{align} \label{Itotransformation}
\bzz_t=\bps(\bxx_t)=\begin{bmatrix}
\bxi \\ 
\bet
\end{bmatrix}
: = \begin{bmatrix}
\diag(\bmm \oslash \bsim) & 0 \\ 
0 & \diag(\bmm \oslash \bsim)
\end{bmatrix}
\begin{bmatrix}
\bth \\ 
\bom 
\end{bmatrix}
\end{align} i.e.
\begin{align}
    \bzz_t&= [\psi_1(X_{1,t}),\dots,\psi_d(X_{d,t}), \psi_{d+1}(X_{d+1,t}), \dots, \psi_{2d}(X_{2d,t})]^{\top} \nonumber \\     & =\left[\frac{m_1}{\sigma_1}\theta_{1,t},\:\dots\:,\frac{m_d}{\sigma_d}\theta_{d,t} ,  \dots , \frac{m_1}{\sigma_1} \omega_{1,t} , \dots,\frac{m_d}{\sigma_d} \omega_{d,t} \right]^{\top}
\end{align}
\end{thm}
\begin{proof}
Since the transformation (\ref{Itotransformation}) is a linear transformation and doesn't depend on $t$, the first and third term in (\ref{Itorule})  vanish so we are left with
\begin{align}
    \diff Z_{k,t}& = \sum_{i=1}^{2d} \frac{\partial \psi_k}{\partial x_i}(\bxx_t) \diff X_{i,t} \nonumber \\ &= \sum_{i=1}^{2d} \frac{\partial \psi_k}{\partial x_i}(X_{k,t}) \diff X_{i,t} \nonumber  \\
    &= \frac{\partial \psi_k}{\partial x_k} \diff X_{k,t} \nonumber \\ 
    & = \begin{cases}
    \dfrac{m_k}{\sigma_k}\diff \theta_{k,t} & 1\leq k \leq d \nonumber \\ 
     \dfrac{m_k}{\sigma_k}\diff \omega_{k,t} & n+1 \leq k \leq 2d \\ 
    \end{cases} \nonumber \\ 
    &=  \begin{cases}
    \dfrac{m_k}{\sigma_k} \omega_{k,t} \diff t & 1\leq k \leq d \nonumber \\ 
  \dfrac{m_k}{\sigma_k}  \left(-\dfrac{\gamma_k}{m_k} \omega_{k,t}- \dfrac{P_k}{\sigma_k}- \displaystyle \sum_{j=1}^d \dfrac{K_{k,j}}{m_k}\sin(\theta_{k,t}-\theta_{j,t})+ \dfrac{\sigma_k}{m_k}\diff w_{k,t}\right) & n+1 \leq k \leq 2d
    \end{cases} \nonumber \\ 
    & = \begin{cases}
    \eta_{k,t} \diff t & 1\leq k \leq d \nonumber \\ 
    -\dfrac{\gamma_k}{m_k} \eta_{k,t}- \dfrac{P_k}{\sigma_k}- \displaystyle \sum_{j=1}^d \dfrac{K_{k,j}}{\sigma_k}\sin\left(\dfrac{\sigma_k}{m_k}\xi_{k,t}-\dfrac{\sigma_j}{m_j}\xi_{j,t}+\varphi_{ij}\right)+\diff w_{k,t} & n+1 \leq k \leq 2d
    \end{cases} 
\end{align}
Writing in vector form, we get 
\begin{align}
\begin{bmatrix}
\diff \bxi \\
  \diff \bet
\end{bmatrix}
= \begin{bmatrix}
 \bet \\ 
      -\nabla_{\bet}F(\bet)- \nabla_{\bxi}U(\bxi) 
\end{bmatrix} \diff t
+
 \begin{bmatrix}
     \bm{0}_{d\times d} & \bm{0}_{d\times d} \\ 
   \bm{0}_{d\times d}&\bm{I}_{d\times d}
    \end{bmatrix} \diff \bm{w}
\end{align}
where $F=\frac{1}{2} \langle \bet,\diag(\bg\oslash \bmm)\bet \rangle$ and the potential function in $\bxi$ is given by
\begin{align}
    U(\bxi) = \sum_{i=1}^{2d}\frac{P_i}{\sigma_i} \xi_i + \sum_{(i,j)\in \mathcal{E}} K_{i,j}\frac{m_i}{\sigma_i^2}\left( 1-\cos\left(\dfrac{\sigma_i}{m_i}\xi_i-\frac{\sigma_j}{m_j}\xi_j + \varphi_{ij}\right)\right)
\end{align}
\end{proof}
The Fokker-Planck equation of this SDE is 
\begin{align}
    \frac{\partial \rho}{\partial t} = -\langle \bet, \nabla_{\bxi} \rho\rangle + \nabla_{\bet} \cdot \left( \left( \nabla _{\bet}F(\bet)  + \nabla_{\bxi} U(\bxi)
    \right) \rho\right) + \frac{1}{2} \Delta_{\bet} \rho
 \end{align}
 
 
 \begin{remark} Given a random vector $X$, who joint pdf is given by $f_{X}(x)$ and if $H:\mathbb{R}^n \mapsto \mathbb{R}^n$ is a $1-1$ differentiable function, then the random vector $Y=H(X)$ also has a density function given by 
 \begin{align}
     f_{Y}(y)= \frac{f_X(x)}{\lvert {\rm{det}} \nabla H(x)\rvert}, 
     \quad x\in H^{-1}(y)
 \end{align}
 \end{remark}
% We claim that 
% \begin{align}
% \rho_{\infty} \propto \exp\left(- \frac{1}{2} \langle \bom, \bM \bom \rangle -V(\theta) \right)
% \end{align}
% is the stationary solution. 
% \\
% \\
% {\color{red} Use rule: $\nabla(\rho F)=(\nabla \rho). F+\rho (\nabla. F)$}


% \begin{align}
%     0 &=- \langle \bom ,\nabla_{\bth}\rho_{\infty} \rangle  + \nabla_{\bom} \cdot \left( \rho_{\infty} \left( \bM^{-1} \Gamma \bom + \bM^{-1}\nabla_{\bth}V(\bth) +\bD \nabla_{\bom} \log \rho_{\infty}  \right) \right) \nonumber \\ 
%     & = \rho_{\infty} \langle \bom, \nabla_{\bth}V(\bth) \rangle  + \langle \nabla_{\bom} \rho_{\infty}, \bM^{-1}\Gamma \bom + \bM^{-1}\nabla_{\bth}V(\bth) +\bD \nabla_{\bom} \log \rho_{\infty} \rangle \nonumber \\ 
% &+ \rho_{\infty} \nabla_{\bom} \cdot(\bM^{-1} \Gamma \bom + D \nabla_{\bom} \log \rho_{\infty }) \nonumber \\ 
% & = \rho_{\infty} \langle \bom, \nabla_{\bth}V \rangle -\rho_{\infty} \langle  \bM \bom ,\bM^{-1}\Gamma \bom \rangle -\rho_{\infty} \langle  \bM \bom ,\bM^{-1}\nabla_{\bth}V(\bth)  \rangle- \rho_{\infty}\langle  \bM \bom ,\bD \bM \omega  \rangle \nonumber \\ 
% &+ \rho_{\infty} \tr(\bM^{-1}\Gamma) - \rho_{\infty} \tr(\bD \bM) \nonumber \\ 
% &= {-}\rho_{\infty} \langle \bom, \Gamma \bom \rangle - \frac{1}{2} \rho_{\infty} \langle \bom, \bSim \bSim^{\top}\bom \rangle  + \rho_{\infty} \tr(\bM^{-1}\Gamma) {-}\rho_{\infty} \tr(\bD \bM) \nonumber \\ 
% & = \rho_{\infty} \tr\left( \bM^{-1} \left( \Gamma-\frac{1}{2}  \bSim \bSim^{\top}\right)+\left( \Gamma -\frac{1}{2} \bSim\bSim^{\top}\right)\bom \bom^{\top}\right)
% \end{align}


% \begin{align}
%     2\Gamma = \bSim \bSim^{\top}
% \end{align}

% The Lyapunov functional for this system is 
% \begin{align}
%   F(\rho)=  \int \rho(\bom,\bth) \log \rho(\bom,\bth) \: \diff \bom \diff \bth  + \int \left( \rho(\bom,\bth) ( \frac{1}{2} \langle \bom, \bM \bom \rangle +   V(\bth) \right)   \: \diff \bom \diff \bth 
% \end{align}


% \begin{align}
% \frac{\diff F}{\diff t} &= \langle \frac{\delta F}{\delta \rho }, \frac{\partial \rho  }{\partial t}\rangle \nonumber \\ 
% & = \langle  \frac{\delta F}{\delta \rho }, -\langle \bom ,\nabla_{\bth}\rho \rangle  + \nabla_{\bom} \cdot \left( \rho \left( \bM^{-1}\bm{\Gamma}\bom + \bM^{-1}\nabla_{\bth}V(\bth) +\bD \nabla_{\bom} \log \rho \right) \right \rangle \nonumber  \\
% & = \int \left ( \log \rho(\bom,\bth) +1 + \frac{1}{2}\langle \bom, \bM \bom \rangle + V(\bth) \right) (-\langle \bom ,\nabla_{\bth}\rho \rangle ) \nonumber\\
% & +\left ( \log \rho(\bom,\bth) +1 + \frac{1}{2}\langle \bom, \bM \bom \rangle + V(\bth) \right)   \nabla_{\bom} \cdot \left(  \rho  \bM^{-1}\bm{\Gamma}\bom  \right. \nonumber\\
% &+ \left. \bM^{-1}\nabla_{\bth}V(\bth) +\bD \nabla_{\bom} \log \rho \right) \diff \bom \diff \bth  \nonumber\\
% &= \int \rho   \nabla_{\bth} \cdot \left ( \bom \left( \log \rho(\bom,\bth) +1 + \frac{1}{2}\langle \bom, \bM \bom \rangle + V(\bth)  \right) \right)  \nonumber\\
% &- \int \nabla_{\bom} \left ( \log \rho  (\bom,\bth) +1 + \frac{1}{2}\langle \bom, \bM \bom \rangle + V(\bth) \right)
%  \cdot \left(  \bM^{-1}\bm{\Gamma}\bom  \right. \nonumber\\
% &+ \left. \bM^{-1}\nabla_{\bth}V(\bth) +\bD \nabla_{\bom} \log \rho \right) \rho  \diff \bom \diff \bth  \nonumber\\
% &= \int \langle \nabla_{\bth} \rho,  \bom \rangle+\rho \langle \nabla_{\bth} V(\bth),  \bom \rangle- 
% \rho \langle \nabla_{\bom} \log \rho,  \bM^{-1} \bGam \bom  \rangle- \rho \langle \bM \bom,  \bM^{-1} \bGam \bom  \rangle \nonumber\\
% &-\rho \langle \nabla_{\bom} \log \rho,  \bM^{-1} \nabla_{\bth}V(\bth) \rangle- \rho \langle \bM \bom,  \bM^{-1} \nabla_{\bth}V(\bth)  \rangle \nonumber \\
% &-
% \rho \langle \nabla_{\bom} \log \rho,  D \nabla_{\bom} \log \rho \rangle- \rho \langle \bM \bom,  D \nabla_{\bom} \log \rho \rangle \nonumber \: \diff \bom \diff \bth \\
% &= \int \rho \langle \nabla_{\bth} V(\bth),  \bom \rangle- 
% \rho \langle \nabla_{\bom} \log \rho,  \bM^{-1} \bGam \bom  \rangle-\rho \langle \bM \bom,  \bM^{-1} \bGam \bom  \rangle \nonumber\\
% &-\rho \langle \bM \bom,  \bM^{-1} \nabla_{\bth}V(\bth)  \rangle 
% -\rho \langle \nabla_{\bom} \log \rho,  D \nabla_{\bom} \log \rho \rangle- \rho \langle \bM \bom,  D \nabla_{\bom} \log \rho \rangle \nonumber \: \diff \bom \diff \bth \\
% &= \int - \rho \left(
%  \langle \nabla_{\bom} \log \rho,  \bM^{-1} \bGam \bom + D\bM \bom \rangle + \langle  \bom,   \bGam \bom  \rangle \right. \nonumber\\
% &+ \left. \langle \nabla_{\bom} \log \rho,  D \nabla_{\bom} \log \rho \rangle \right )  \: \diff \bom \diff \bth \nonumber\\
% &=-\mathbb{E}_{\rho} \{ \lVert \nabla_{\bom} \log \rho+\bM \bom \rVert_{D}^{2} \} \leq 0
% \end{align}


  
 



% The function $L(\bom,\bth) = \frac{1}{2} \langle \bom , M \bom \rangle  + V(\bth)$ is a Lyapunov for the determinstic system i.e. no noise. 

% \begin{align}
%     \dot{L} &= \langle \nabla_{\bth} L , \dot{\bth} \rangle + \langle \nabla_{\bom} L , \dot{\bom} \rangle \nonumber  \\ 
%             & = \langle \nabla_{\bth} ,(\bth), \bom \rangle + \langle \bM \bom, -\bM^{-1} \bGam \bom - \bM^{-1}\nabla_{\bth} V(\bth)\rangle \nonumber  \\ 
%             & = - \langle \bom, \bGam \bom \rangle  \leq 0
% \end{align}




% \begin{thm}[Lamperti Transform] \label{LampertiTransform} Let $X_t$ be the solution to the (Ito) SDE 
% \begin{align}
%     \diff \bxx_t = \bbf(\bxx_t,t) \: 
%     \diff t + \bbeta(\bxx_t,t) \: \bR(t)\diff \bm{W}_t
% \end{align}
% where $\bxx_t,w_t \in \mathbb{R}^d$, $\bR(t)\in \mathbb{R}^{d \times d}$ is any matrix function of $t$ and $\bbeta \in \mathbb{R}^{d \times d} $ is a diagonal matrix whose diagonal elements which we denote 
% \begin{align*}
%     \beta_{i,i}(\bxx_t.t)= \beta_i(X_{i,t},t),
% \end{align*}
% i.e., each diagonal element depend only on the $i$th component of $\bxx_t$. Then, the transformation defined by 
% \begin{align} \label{ChangeofVariable}
% Z_{i,t} = \psi_i(X_{i,t},t) = \int \frac{1}{\beta_i(x,t)} \: \diff x \bigg |_{x=X_{i,t}}
% \end{align}
% will result in the diffusion process 
% \begin{align} \label{CoVDiffusion}
%     \diff Z_{i,t} = \left(\frac{\partial}{\partial t}\psi_i(x,t)\bigg|_{x=\psi^{-1}(Z_{i,t},t)} + \frac{f_i(\bps^{-1}(\bzz,t),tt)}{\beta_i(\psi_i^{-1}(Z_{i,t},t),t)} +\frac{1}{2}  \frac{\partial}{\partial x}
%     \beta_i(\psi_i^{-1}(Z_{i,t},t) \right)\diff t + \sum_{j=1}^{d}
%  r_{i,j}(t)\: \diff w_{j,t}
% \end{align}
% where $r_{i,j}(t)$ are the elements of $\bR(t)$ and $\bxx_t = \psi^{-1}(\bzz,t)$.
% \end{thm}

% \begin{proposition}
% Let $d=2n$ and set 
% \begin{align} \label{partitionvector}
%      \bxx_t = \begin{bmatrix}\bth \\ 
%     \bom
%     \end{bmatrix}, \quad  \bzz_t := \begin{bmatrix}
%     \bxi \\
%     \bet 
%     \end{bmatrix}.
% \end{align}
% Let 
% \begin{align}
%     \bbeta(\bxx_t,t) \equiv   \begin{bmatrix}
%     \bI & 0 \\ 
%     0 & \diag(\bsim \oslash \bmm )
%     \end{bmatrix} ,
%     \quad 
%     \bR(t) \equiv 
%      \begin{bmatrix}
%     0 & 0 \\ 
%     0 & \bI
%     \end{bmatrix} 
% \end{align}
% then applying Theorem \ref{LampertiTransform} to (\ref{ODEsystem}) results in the diffusion process
% \begin{align}
%     \diff \bz_t
% \end{align}
% \end{proposition}
% \begin{proof}
% Notice that performing the change of variable (\ref{ChangeofVariable})
% results in 
% \begin{align}
%     \bzz_t = \bps(\bxx_t,t) = 
%      \begin{bmatrix}
%     \bI & 0 \\ 
%     0 & \diag(\bmm \oslash \bsim )
%     \end{bmatrix} \bxx_t
% \end{align}
% so that 
% \begin{align}
%     \bxx_t = \bps^{-1}(\bzz_t,t) =
%       \begin{bmatrix}
%     \bI & 0 \\ 
%     0 & \diag(\bsim \oslash \bmm )
%     \end{bmatrix}
%     \bzz_t.
% \end{align}
% Using (\ref{partitionvector}), the inverse transformation can be written component wise as 
% \begin{align} \label{compontentwisetransform}
%     \theta_i = \xi_i,\quad \omega_i = \frac{\sigma_i}{m_i}  \eta_i
% \end{align}
% Notice that since $\bps$ and $\bbeta$ are independent of $t$ and $x$, then the first and third term in the drift of (\ref{CoVDiffusion}) vanish and we are left with the second term. Notice that we have
% \begin{align}
%   \frac{f_i(\bxx_t,t)}{\beta_i(\bxx_{i,t},t)} = \begin{cases}
%   \omega_i & 1\leq i \leq n \\ 
%   -\frac{\gamma_i}{m_i}\frac{m_i}{\sigma_i} \omega_i -\frac{m_i}{\sigma_i}\frac{P_i}{m_i} - \displaystyle \sum_{j=1}^{n} \frac{m_i}{\sigma_i} \frac{1}{m_i}k_{ij} \sin(\theta_i-\theta_j) & n+1\leq i \leq 2n
%   \end{cases}
% \end{align}
% which means that 
% \begin{align}
%     \frac{f_i(\bps^{-1}(\bzz_T,t),t)}{\beta_i(\psi^{-1}(Z_{i,t},t),t)} = \begin{cases}
%     \frac{\sigma_i}{m_i} \eta _i & 1\leq i \leq n \\ 
%     -\frac{\gamma_i}{m_i}\eta_i-\frac{P_i}{\sigma_i} -\displaystyle \sum_{j=1}^{n}\frac{1}{\sigma_i} k_{i,j}\sin(\xi_i-\xi_j) & n+1\leq i \leq 2n
%     \end{cases}.
% \end{align}
% This results in the diffusion process 
% \begin{align} \label{LampertiSDE}
% \diff \begin{bmatrix}
%  \bxi \\ \bet 
% \end{bmatrix}
% &= 
% \begin{bmatrix} 
% \diag(\bsim \oslash \bmm) \bet \\
% - \diag(\bg \oslash \bmm) \bet- \nabla_{\bxi} U(\bxi) 
%   \end{bmatrix}
%   \: \diff t
%   + \begin{bmatrix}
%     0 & 0 \\ 
%     0 & \bI
%     \end{bmatrix} \: \diff \bm{W}_t \\ 
%     :&=\begin{bmatrix}
%  \bm{G}_1 \bet \\
%  -\nabla _{\bet} F(\bet)  - \nabla_{\bxi} U(\bxi)
% \end{bmatrix} \: \diff t   + \begin{bmatrix}
%     0 & 0 \\ 
%     0 & \bI
%     \end{bmatrix} \: \diff \bm{W}_t
% \end{align}
% where $F(\bet) =\frac{1}{2} \bet^{\top} \bm{G}_2 \bet$
% where the potential function $U$ is given by 
% \begin{align}
% U(\bxi):= \sum_{i=1}^{n} \frac{P_i}{\sigma_i} \xi_i + \sum_{i,j\in \mathcal{E}}\frac{1}{\sigma_i} k_{i,j}(1-\cos(\xi_i-\xi_j))\end{align}
% \end{proof}




% % %\subsection{Stationary Solution}
% % 5To compute the stationary solution, we set 
% % \begin{align} \label{EqualZeroPDE}
% %     0 = \langle \bm{G}_1\bet, \nabla_{\bxi} \rho\rangle + \nabla_{\bet} \cdot \left( \left( \nabla _{\bet}F(\bet)  + \nabla_{\bxi} U(\bxi)
% %     \right) \rho\right) + \Delta_{\bet} \rho
% % \end{align}
% % and we have the ansatz 
% % \begin{align}
% %     \rho_{\infty} \propto \exp\left(-(F(\bet)+U(\bxi)\right).
% % \end{align}
% % From this, we get the derivatives 
% % \begin{align} \label{derivatives}
% %     \nabla_{\bxi} \rho_{\infty} = -\pinf \nabla_{\bxi}U(\bxi) ,
% %     \quad \nabla_{\bet} \pinf = -\pinf \nabla_{\bet} F(\bet).
% % \end{align}
% % Using (\ref{derivatives}), we obtain 
% % \begin{align}
% %     \Delta_{\eta}\pinf = \pinf \lVert G_2 \eta \rVert^2 - \pinf {\rm{tr}}(G_2) 
% % \end{align}
% % and 
% % \begin{align}
% %       \nabla_{\bet} \cdot \left( \left( \nabla _{\bet}F(\bet)  + \nabla_{\bxi} U(\bxi)
% %     \right)\right)= -\pinf \lVert G_2 \eta \rVert^2 - \pinf \langle G_2\bet , \nabla_{\bxi}U(\bxi) + \pinf {\rm{tr}}(G_2)
% % \end{align}t
% % which implies that (\ref{EqualZeroPDE}) reduces to 
% % \begin{align}
% %     0 = \pinf \langle (G_2-G_1)\bet, \nabla_{\bxi} U(\bxi) \rangle.
% % \end{align}
% % In order for this to hold for all $\bet,\bxi$ we must have $G_2=G_1$ which means that $\bsim=\bg$, i.e., $\sigma_i=\gamma_i$ for all $i$.

% \subsection{Free Energy}
% Consider the functional 
% \begin{align}
%     \Phi(\rho)& =\int \rho(\bet,\bxi) \log  \rho(\bet,\bxi) \: \diff \bet \diff \bxi  + \beta
%     \int \rho(\bet,\bxi) U(\bxi)\: \diff \bet \diff \bxi  
%     +
%   \beta \int \rho(\bet,\bxi) F(\bet) \: \diff \bet \diff \bxi \nonumber \\ 
%   & = \beta \widetilde{\Phi}(\rho) 
% \end{align}
% We compute 
% \begin{align}
%     \frac{\diff \Phi}{\diff t}&=\int \frac{\partial \rho}{\partial t} \left( \log \rho + \beta F +  \beta U\right)\: \diff \bet \diff \bxi \: \diff \bet \diff \bxi    \nonumber \\
%     &  = \int \nabla_{\bet} \cdot \left( \rho \left( \nabla_{\bet} \log \rho+ \nabla_{\bet} F + \nabla_{\bxi }U\right)\right) \left( \log \rho + \beta F +  \beta U\right) \: \diff \bet \diff \bxi   \nonumber \\ 
%     & \: \: -  \int \langle G_1\bet , \nabla_{\bxi} \rho \rangle \left( \log \rho + \beta F + \beta U\right)\: \diff \bet \diff \bxi   \nonumber \\ 
%     & = \int - \lvert \nabla_{\bet} \log \rho \lvert^2\rho  -( \beta+1) \langle \nabla_{\bet} \log \rho ,\nabla_{\eta}F\rangle\rho    - \beta \lvert \nabla_{\bet} F \rvert^2\rho \: \diff \bet \diff \bxi   \nonumber \\ 
%     &  \int -\langle \nabla_{\bxi} U , \nabla_{
%     \bet} \log \rho \rangle\rho  -\beta \langle \nabla_{\bxi} U , \nabla_{
%     \bet} F \rangle\rho  + \langle G_1 \bet, \nabla_{\bxi} \log \rho\rangle \rho  + \beta  \langle G_1 \bet, \nabla_{\bxi} U \rangle\rho  \: \diff \bet \diff \bxi   \nonumber  \\
%     & = \int - \lvert \nabla_{\bet} \log \rho \lvert^2\rho  - (\beta+1) \langle \nabla_{\bet} \log \rho ,\nabla_{\eta}F\rangle\rho    - \beta \lvert \nabla_{\bet} F \rvert^2\rho \: \diff \bet \diff \bxi   \nonumber \\
%     &+ \int -\beta \langle \nabla_{\bxi} U , \nabla_{
%     \bet} F \rangle\rho +  \beta  \langle G_1 \bet, \nabla_{\bxi} U \rangle\rho. \nonumber 
% \end{align}
% A sufficient condition for $  \dfrac{\diff \Phi}{\diff t}\leq 0$ is to set 1 $G_!\bet =\nabla_{\bet} F = G_2 \bet \Rightarrow 
% \bsim = \bg \Rightarrow \sigma_i=\gamma_i$ for all $i$.


% Let $L$ be a Lyapunov function for the system 
% \begin{align}
% \begin{bmatrix}
% \dot{\bxi} \\ \dot{\bet}
% \end{bmatrix}
% =\begin{bmatrix}
%  \bm{G}_1 \bet \\
%  -\nabla _{\bet} F(\bet)  - \nabla_{\bxi} U(\bxi)
% \end{bmatrix} 
% \end{align}
% that is $L$ satisfies 
% \begin{align} \label{lyapcondition}
%     \dot{L}(\bxi,\bet) &= \nabla_{\bxi} L \dot{\bxi} + \nabla_{\bet} L\dot{\bet} =\nonumber\\ 
%     &=\langle \nabla_{\bxi}L,\bG_1 \bet \rangle - 
%     \langle \nabla_{\bet} L, \nabla _{\bet} F(\bet)  + \nabla_{\bxi} U(\bxi) \rangle \nonumber
%     \\ 
%     & =\langle \nabla_{\bxi}L,\bG_1 \bet \rangle 
%      -\langle \nabla_{\bet} L, \nabla _{\bet} F(\bet)  \rangle - \langle \nabla_{\bet} L, \nabla_{\bxi} U(\bxi) \rangle
%     \leq -W(\
% \end{align} 
% If $L=F+U$
% \begin{align}
%     \dot{L}(\bxi,\bet) &= \langle \nabla_{\bxi}U,\bG_1 \bet \rangle 
%      -\langle \nabla_{\bet} F, \nabla _{\bet} F(\bet)  \rangle - \langle \nabla_{\bet} F, \nabla_{\bxi} U(\bxi) \rangle \nonumber \\
%      & =\langle \nabla_{\bxi}U,\bG_1 \bet \rangle - \langle \bG_2 \bet , \bG_2\bet  \rangle - \langle \bG_2 \eta ,\nabla_{\bxi}U \rangle \nonumber \\ 
%      & = - \lVert \bG_2 \bet \rVert^2 + \langle \nabla_{\bxi} U, (\bG_1-\bG_2) \bet  \rangle
% \end{align}
% We compute 
% \begin{align}
%     \frac{\diff \Phi}{\diff t}&=\int \frac{\partial \rho}{\partial t} \left( \log \rho + L\right)\: \diff \bet \diff \bxi \:   \nonumber \\
%     &  = \int \nabla_{\bet} \cdot \left( \rho \left( \nabla_{\bet} \log \rho+ \nabla_{\bet} F + \nabla_{\bxi }U\right)\right) \left( \log \rho + L\right) \: \diff \bet \diff \bxi   \nonumber \\ 
%     & \: \: -  \int \langle G_1\bet , \nabla_{\bxi} \rho \rangle \left( \log \rho + L\right)\: \diff \bet \diff \bxi   \nonumber \\ 
%     & = \int - \lvert \nabla_{\bet} \log \rho \lvert^2\rho  - \langle \nabla_{\bet} \log \rho ,\nabla_{\eta}L\rangle\rho    - \langle \nabla_{\bet} \log \rho ,\nabla_{\eta}F\rangle\rho -\langle \nabla_{\bet} F,\nabla_{\eta}L\rangle\rho \: \diff \bet \diff \bxi   \nonumber \\ 
%     &  \int -\langle \nabla_{\bxi} U , \nabla_{
%     \bet} \log \rho \rangle\rho  -\langle \nabla_{\bxi} U , \nabla_{
%     \bet} L \rangle\rho  + \langle G_1 \bet, \nabla_{\bxi} \log \rho\rangle \rho  +  \langle G_1 \bet, \nabla_{\bxi} U \rangle\rho  \: \diff \bet \diff \bxi   \nonumber  \\
%     & = \int - \lvert \nabla_{\bet} \log \rho \lvert^2\rho    - \langle \nabla_{\bet} \log \rho ,\nabla_{\eta}L\rangle\rho    - \langle \nabla_{\bet} \log \rho ,\nabla_{\eta}F\rangle\rho  \: \diff \bet \diff \bxi   \nonumber \\
%     &+  \int - \langle  \nabla_{
%     \bet} L ,\nabla_{\bxi} U\rangle\rho +    \langle \nabla_{\bxi} L, G_1 \bet \rangle\rho -\langle \nabla_{\eta}L, \nabla_{\bet} F\rangle\rho \: \diff \bet \diff \bxi  \nonumber  \\ 
%   %\leq & \int - \lvert \nabla_{\bet} \log \rho \lvert^2\rho    - \langle \nabla_{\bet} \log \rho ,\nabla_{\eta}L\rangle\rho    - \langle \nabla_{\bet} \log \rho ,\nabla_{\eta}F\rangle\rho \: \diff \bet \diff \bxi   \nonumber  
%   &= \mathbb{E}[\dot{L}] - \mathbb{E}\left[\lvert \nabla_{\bet} \log \rho \rvert^2 + 2\langle \nabla_{\bet} \log \rho, \nabla_{\bet} F\rangle  \right]
% \end{align}
% If $\rho = \exp(-F+H)$ then $\log \rho  = -F + H$ 






% where the inequality follows from (\ref{lyapcondition}). Provided that
% $\nabla_{\bet} L =- \nabla_{\bet} F $ almost everywhere, we should have 
% $\frac{\diff \Phi}{\diff t} \leq 0$.
% \begin{align}
%     L(\bxi,\bet) = L_0 + F(\bet)
% \end{align}
 \end{document}

