\documentclass[a4paper,11pt]{article}

\usepackage{amsmath,amsfonts,amssymb,amsthm}
\usepackage{color}
\usepackage[top=1in,bottom=1in,margin=1in]{geometry}

\title{Structure Preserving Power Systems Model}
% sort your names alphabetically by last name
\author{
  Abhishek Halder
}
\date{} % change this accordingly

\begin{document}

\maketitle

\subsection*{Dynamics}
We consider a power network with $n>1$ generators, where the $i$th generator's dynamics is modeled as
\begin{subequations}
\begin{align}
   \dot{\theta}_{i} &= \omega_{i},\\
   m_{i}\dot{\omega}_{i} &= P_{i} - \gamma_{i}\omega_{i} - \displaystyle\sum_{j=1}^{n}k_{ij}\sin(\theta_{i}-\theta_{j} - \varphi_{ij}).
\end{align}
\label{GenDyn}
\end{subequations}
The state variables are the \textbf{rotor angles} $\theta_{i}\in\mathbb{S}^{1}\equiv [0,2\pi)$, and the \textbf{rotor angular velocities} $\omega_{i}\in\mathbb{R}$ for $i=1,...,n$. Recall that the $n$-torus $\mathbb{T}^{n} = \underbrace{\mathbb{S}^{1} \times \hdots \times \mathbb{S}^{1}}_{n \; \text{times}}$. Thus the state space for (\ref{GenDyn}) is $\mathbb{T}^{n}\times\mathbb{R}^{n}$.

\subsection*{Parameters}
The parameters $m_{i}>0,\gamma_{i}>0$ respectively denote the inertia and damping coefficient for the $i$th generator. 

The other three parameters: the \textbf{effective power input} $P_{i}$, the \textbf{phase shift} $\varphi_{ij}\in[0,\frac{\pi}{2})$, and the \textbf{coupling} $k_{ij}\geq 0$, depend on the network reduced admittance matrix $Y\in\mathbb{C}^{n\times n}$. 

Specifically,
\begin{subequations}
\begin{align}
 P_{i} &= P_{i}^{\text{mech}} - E_{i}^{2} \Re\left(Y_{ii}\right), \\
 \varphi_{ij} &= \begin{cases} -\arctan\left(\dfrac{\Re\left(Y_{ij}\right)}{\Im\left(Y_{ij}\right)}\right), & \text{if}\;i\neq j,\\
 0, & \text{otherwise},
 \end{cases}\\
 k_{ij} &= \begin{cases} E_{i}E_{j} \vert Y_{ij}\vert, & \text{if}\;i\neq j,\\
 0, & \text{otherwise},
 \end{cases}
\end{align}
\end{subequations}
where $P_{i}^{\text{mech}}$ is the mechanical power input, and $E_{i}$ is the internal voltage (magnitude) for generator $i$.

\subsection*{What we need}
The IEEE 14 bus has 5 generators. We want $m_i, \gamma_i, P_{i}^{\text{mech}}, E_{i}$ and the $5\times 5$ admittance matrix $Y$ for the IEEE 14 bus test system.


\section{Generator Parameters} % (fold)
\label{sec:generator_parameters}
Important notes:
\begin{itemize}
\item \textbf{angles $\theta$ and rotor speeds $\omega$ must be in rad (NOT deg).}
\item \textbf{everything is in per unit (p.u)}. For example, $P_{\text{elec}} = 0.8$p.u. with $\text{baseMVA}=100$ means $P_{\text{elec}} = 80$MVA.
\end{itemize}
\textcolor{red}{Question: where is the voltage magnitude $|V_i|$ of bus $i$? \eqref{GenDyn} only has angles.}

GENCLS Model:
\begin{subequations}
\begin{align}
& \dot{\theta}  = \omega_0 \omega \\
& \dot{\omega} = \frac{1}{2H}(P_{\text{mech}} - D \omega - P_{\text{elec}})
\end{align}
\end{subequations}
Comparing GENCLS model with \eqref{GenDyn}:
\begin{subequations}
\begin{align}
& m = 2\omega_0 H \\
& \gamma = \omega_0 D
\end{align}
\end{subequations}
\begin{table}[tb]
	\caption{Generator Parameters for GENCLS Model ($\omega_0 = 2\pi\times 60\text{Hz}$)}
	\label{tab:GENCLS_Parameters}
	\centering

	\begin{tabular}{c|c|cccc}
	\hline

	\hline
	\textbf{Generator} & \textbf{Bus} & $H$ & $D$ & $m = 2\omega_0 H$ & $\gamma = \omega_0 D$ \\
	\hline
	1	 & 1 & 2.64 & 4.00 & $1.9905 \times 10^3$ & $1.5080 \times 10^3$ \\
	2	 & 2 & 2.61 & 4.00 & $1.9679 \times 10^3$ & $1.5080 \times 10^3$ \\
	3	 & 3 & 3.42 & 4.00 & $2.5786 \times 10^3$ & $1.5080 \times 10^3$ \\
	4	 & 6 & 2.45 & 4.00 & $1.8473 \times 10^3$ & $1.5080 \times 10^3$ \\
	5	 & 8 & 3.59 & 4.00 & $2.7068 \times 10^3$ & $1.5080 \times 10^3$ \\
	\hline

	\hline
	\end{tabular}
\end{table}
Other parameters are in CSV files
\begin{description}
\item[case14.m] Matpower casefile,
\item[get\_parameters\_case14.m] Matlab script to get all scripts and write into csv files
\item[gen\_parameters.csv] everything in Table \ref{tab:GENCLS_Parameters}
\item[bus-init.csv] about initial values of $\theta_i$
\item[line-parameters.csv] everything about lines, $i,j,k_{ij},\phi_{ij}$
\end{description}
When simulating line failures, you can remove any one of 20 lines of the 14-bus system (any line in line-parameters.csv). 

\section{Load Modeling} % (fold)
\label{sec:load_modeling}
The most commonly used load modeling is ZIP load:
\begin{itemize}
\item Z: constant impedance Z
\item I: constant current I
\item P: constant power P
\end{itemize}
So given the voltage at bus $i$, the real load is calculated as
\begin{equation}
P^{\text{load}}_i = P_i + \text{Real}(V_i I_i^\dagger) + \text{Real}(V_i * (V_i/Z_i)^\dagger)
\end{equation}
Note we are ignoring reactive load (imaginary parts) in the swing equation.

\subsection{Load Models in Kron Reduction} % (fold)
In Florian's slides, they are using constant current and impedance loads (see Figure on page 8/41).
\begin{itemize}
\item the constant impedance part is modeled as self-loops in Kron reduction (see Page 33/41), then integrated in the $Y_{red}$ matrix
\item the constant current load can be considered in the Kron reduction, more details below
\item I believe Kron reduction has trouble dealing with constant power load, so Florian (and us) don't consider it (for load buses).
\end{itemize}

% subsection constant_current_load (end)

\subsection{Constant Current Load} % (fold)
\label{sub:constant_current_load}
Assume all load at non-generator buses are modeled as constant current load, with current $I_i$ $i \in \mathcal{L}$. 

Original equations with constant current load
\begin{subequations}
\begin{eqnarray}
M_i \ddot{\theta} + D_i \dot{\theta_i} = P_{mech,i} - P_{load,i} - \sum_{j} |V_i| |V_j| |Y_{ij}| \sin(\theta_i - \theta_j + \phi_{ij}) &~ i \in \mathcal{G}~\text{buses with generators} \\
0 = - \text{Real}(V_i I_i^\dagger) - \sum_{j} |V_i| |V_j| |Y_{ij}| \sin(\theta_i - \theta_j + \phi_{ij}) &~ i \in \mathcal{L}~\text{buses without generators} \\
\end{eqnarray}
\end{subequations}
Note: $P_{mech,i} \in \mathbb{R}$, $P_{load,i} \in \mathbb{R}$ and $I_i \in \mathbb{C}$ are all give parameters; load current is a complex number. 

After Kron reduction
\begin{eqnarray}
Y_{red} &=& Y_{\mathcal{G}\mathcal{G}} - Y_{\mathcal{G}\mathcal{L}} Y_{\mathcal{L}\mathcal{L}}^{-1} Y_{\mathcal{L}\mathcal{G}} \\
I_{red} &=& Y_{\mathcal{G}\mathcal{L}}Y_{\mathcal{L}\mathcal{L}}^{-1} I
\end{eqnarray}

\begin{eqnarray}
M_i \ddot{\theta} + D_i \dot{\theta_i} = P_i - \sum_{j=1, j\ne i}^n |V_i| |V_j| |Y_{red,ij}| \sin(\theta_i - \theta_j + \phi_{red,ij}) - Y_{},\\
~i \in \mathcal{G}~\text{buses with generators}
\end{eqnarray}
where 
\begin{eqnarray}
P_i = P_{mech,i} - P_{load,i} - |V_i|^2 \text{Real}(Y_{red,ii}) + |V_i| |I_{red,i}| \cos(\theta_i - \alpha_i)
\end{eqnarray}

These files/parameters are given
\begin{description}
\item[gen\_parameters.csv] everything in Table \ref{tab:GENCLS_Parameters}
\item[bus-init.csv] about initial values of $\theta_i$
\item[line-parameters.csv] everything about lines, $i,j,k_{ij},\phi_{ij}$
\end{description}
Additional:
\begin{description}
\item[case14.m] Matpower casefile,
\item[get\_parameters\_case14.m] Matlab script to get all scripts and write into csv files
\end{description}
% subsection constant_current_load (end)


% section load_modeling (end)

\end{document}
